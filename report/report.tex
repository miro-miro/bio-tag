%%%%%%%%%%%%%%%%%%%%%%%%%%%%%%%%%%%%%%%%%
% Arsclassica Article
% LaTeX Template
% Version 1.1 (1/8/17)
%
% This template has been downloaded from:
% http://www.LaTeXTemplates.com
%
% Original author:
% Lorenzo Pantieri (http://www.lorenzopantieri.net) with extensive modifications by:
% Vel (vel@latextemplates.com)
%
% License:
% CC BY-NC-SA 3.0 (http://creativecommons.org/licenses/by-nc-sa/3.0/)
%
%%%%%%%%%%%%%%%%%%%%%%%%%%%%%%%%%%%%%%%%%

%----------------------------------------------------------------------------------------
%	PACKAGES AND OTHER DOCUMENT CONFIGURATIONS
%----------------------------------------------------------------------------------------

\documentclass[
11pt, % Main document font size
a4paper, % Paper type, use 'letterpaper' for US Letter paper
oneside, % One page layout (no page indentation)
%twoside, % Two page layout (page indentation for binding and different headers)
headinclude,footinclude, % Extra spacing for the header and footer
BCOR5mm, % Binding correction
]{scrartcl}

%%%%%%%%%%%%%%%%%%%%%%%%%%%%%%%%%%%%%%%%%
% Arsclassica Article
% Structure Specification File
%
% This file has been downloaded from:
% http://www.LaTeXTemplates.com
%
% Original author:
% Lorenzo Pantieri (http://www.lorenzopantieri.net) with extensive modifications by:
% Vel (vel@latextemplates.com)
%
% License:
% CC BY-NC-SA 3.0 (http://creativecommons.org/licenses/by-nc-sa/3.0/)
%
%%%%%%%%%%%%%%%%%%%%%%%%%%%%%%%%%%%%%%%%%

%----------------------------------------------------------------------------------------
%	REQUIRED PACKAGES
%----------------------------------------------------------------------------------------

\usepackage[
nochapters, % Turn off chapters since this is an article        
beramono, % Use the Bera Mono font for monospaced text (\texttt)
eulermath,% Use the Euler font for mathematics
pdfspacing, % Makes use of pdftex’ letter spacing capabilities via the microtype package
dottedtoc % Dotted lines leading to the page numbers in the table of contents
]{classicthesis} % The layout is based on the Classic Thesis style

\usepackage{arsclassica} % Modifies the Classic Thesis package

\usepackage[T1]{fontenc} % Use 8-bit encoding that has 256 glyphs

\usepackage[utf8]{inputenc} % Required for including letters with accents

\usepackage{graphicx} % Required for including images
\graphicspath{{Figures/}} % Set the default folder for images

\usepackage{enumitem} % Required for manipulating the whitespace between and within lists

\usepackage{lipsum} % Used for inserting dummy 'Lorem ipsum' text into the template

\usepackage{subfig} % Required for creating figures with multiple parts (subfigures)

\usepackage{amsmath,amssymb,amsthm} % For including math equations, theorems, symbols, etc

\usepackage{varioref} % More descriptive referencing

%----------------------------------------------------------------------------------------
%	THEOREM STYLES
%---------------------------------------------------------------------------------------

\theoremstyle{definition} % Define theorem styles here based on the definition style (used for definitions and examples)
\newtheorem{definition}{Definition}

\theoremstyle{plain} % Define theorem styles here based on the plain style (used for theorems, lemmas, propositions)
\newtheorem{theorem}{Theorem}

\theoremstyle{remark} % Define theorem styles here based on the remark style (used for remarks and notes)

%----------------------------------------------------------------------------------------
%	HYPERLINKS
%---------------------------------------------------------------------------------------

\hypersetup{
%draft, % Uncomment to remove all links (useful for printing in black and white)
colorlinks=true, breaklinks=true, bookmarks=true,bookmarksnumbered,
urlcolor=webbrown, linkcolor=RoyalBlue, citecolor=webgreen, % Link colors
pdftitle={}, % PDF title
pdfauthor={\textcopyright}, % PDF Author
pdfsubject={}, % PDF Subject
pdfkeywords={}, % PDF Keywords
pdfcreator={pdfLaTeX}, % PDF Creator
pdfproducer={LaTeX with hyperref and ClassicThesis} % PDF producer
} % Include the structure.tex file which specified the document structure and layout

\hyphenation{Fortran hy-phen-ation} % Specify custom hyphenation points in words with dashes where you would like hyphenation to occur, or alternatively, don't put any dashes in a word to stop hyphenation altogether

%----------------------------------------------------------------------------------------
%	TITLE AND AUTHOR(S)
%----------------------------------------------------------------------------------------

\title{\normalfont\spacedallcaps{Designing a Molecular 2-Tag System}} % The article title

%\subtitle{Subtitle} % Uncomment to display a subtitle

\author{\spacedlowsmallcaps{Nicolas Ochsner, Miroslav Phan}} % The article author(s) - author affiliations need to be specified in the AUTHOR AFFILIATIONS block

\date{} % An optional date to appear under the author(s)

%----------------------------------------------------------------------------------------

\begin{document}

%----------------------------------------------------------------------------------------
%	HEADERS
%----------------------------------------------------------------------------------------

\renewcommand{\sectionmark}[1]{\markright{\spacedlowsmallcaps{#1}}} % The header for all pages (oneside) or for even pages (twoside)
%\renewcommand{\subsectionmark}[1]{\markright{\thesubsection~#1}} % Uncomment when using the twoside option - this modifies the header on odd pages
\lehead{\mbox{\llap{\small\thepage\kern1em\color{halfgray} \vline}\color{halfgray}\hspace{0.5em}\rightmark\hfil}} % The header style

\pagestyle{scrheadings} % Enable the headers specified in this block

\tikzset{mytext/.style={rectangle, draw=none, align=center}}

%----------------------------------------------------------------------------------------
%	TABLE OF CONTENTS & LISTS OF FIGURES AND TABLES
%----------------------------------------------------------------------------------------

\maketitle % Print the title/author/date block

\setcounter{tocdepth}{2} % Set the depth of the table of contents to show sections and subsections only

\tableofcontents % Print the table of contents

% \listoffigures % Print the list of figures

% \listoftables % Print the list of tables

%----------------------------------------------------------------------------------------
%	ABSTRACT
%----------------------------------------------------------------------------------------

\section*{Abstract} % This section will not appear in the table of contents due to the star (\section*)

Models of computation are theoretical models, that are used to prove what kind
of computations are possible. One of the most studied models is the universal
Turing machine, which largely defines what computation means nowadays. Another
model, the 2 tag system, can be used to emulate a universal Turing machine. It
works on a string by removing symbols from the front and conditionally appending
symbols at the end. Design of a 2 tag system as a biomolecular process allows us
to compute any problem that can be solved by a universal Turing machine. We
propose an implementation which uses DNA ligation and restriction enzymes to
compute a program that is implemented in a set of engineered sequences.

%----------------------------------------------------------------------------------------
%	AUTHOR AFFILIATIONS
%----------------------------------------------------------------------------------------

%\let\thefootnote\relax\footnotetext{* \textit{Department of Biology, University of Examples, London, United Kingdom}}

%\let\thefootnote\relax\footnotetext{\textsuperscript{1} \textit{Department of Chemistry, University of Examples, London, United Kingdom}}

%----------------------------------------------------------------------------------------

\newpage % Start the article content on the second page, remove this if you have a longer abstract that goes onto the second page

%----------------------------------------------------------------------------------------
%	INTRODUCTION
%----------------------------------------------------------------------------------------

\section{Introduction}

A $m$ tag system is a computational model, that operates on a string of symbols.
At each step it reads the first symbol, removes $m$ symbols from the head of the
string and then appends a sequence of symbols based on the symbol it read to the
string. Mathematically the tag system can be described by the number of symbols
that are removed $m$, the alphabet $\Sigma$ of symbols used and the production
rules, $\Sigma \to \Sigma*$.

% 2 powers of 2 example here.

In the following sections we will first propose a general outline of an
implementation of a 2 tag system and discuss it's features and limitations based
on the processes involved and earlier iterations (namely the iteration we
presented during the last lecture).

\section{Biological Tag System}

\begin{note}
  Unfortunately it turns out that the design of a biological tag system is not
  very easy. (Who would have thought...) However as we still consider it quite
  an interesting idea, we decided to design one for the purpose of discussing
  the solution, its problems and workarounds and the requirements to actually
  make it work.
\end{note}

As an alphabet we use short sequences of DNA which encode a single symbol each.
Our input consists of a sequence of these encodings, where the first symbol
forms a sticky end and the last symbol is also partially exposed as a sticky
end. This enables recognition of the first symbol and appending to the end by
formation of plasmids.

The program is encoded through a set of dsDNA sequences a sticky end on one end
for symbol recognition. These dsDNA sequences encode the production rules, i.e.
they encode, the mapping from each symbol to the string that is supposed to be
appended. The production rules consist of four parts.

\begin{enumerate}
  \item Recognition site: A sticky end consisting of four nucleotides encoding
    the symbol to be recognized.
  \item Restriction target: Two restriction enzyme targets. The first one
    deletes two symbols and opens a sticky end for recognition of the next
    read out. The second one prepares a sticky end on the current string to
    append the next tag.
  \item Production sequence: The sequence of symbols or tag, that needs to be
    appended. It only depends on the recognition site.
  \item Terminal site: A unique terminal sequence of length four that can be cut
    into a sticky end and support appending a sequence.
\end{enumerate}

The recognition site is a 5' overhang of length 4 and can only bind to a
specific symbol on the current working string. 
After the recognition site ligates to the production rule sequence, CRISPR/Cpf1
with a corresponding guide RNA can be used to detect the ligation and open the
sticky end of the terminal site.

Subsequently the terminal site can ligate which leads to the formation of a
plasmid. After knockdown of the guide RNA, we can activate the restriction
enzymes. The first one, EciI, will remove 10 nucleotides and open a sticky end
of length 4 and a 5'-overhang. Two of the nucleotides can be uniquely chosen for
a symbol, while the other two are part of the PAM.

\subsection{Restriction Enzymes}

The constraints, restriction enzymes impose on the design of this system, are
very strong. Most Type IIS restriction enzymes do not cut sufficiently far away
to remove multiple symbols, while still maintaining PAMs or other spacer
sequences inbetween the symbols. Given that the CRISPR/Cpf1 complex creates
overhangs of length 4, we actually need to create spacer sequence of length 4,
which is the same between all symbols. Thus distance from the target size we
require the restriction enzyme to cut is $4 + x + 4 + x$, where $x$ is the
number of nucleotides we use for encoding the symbols.

Restriction enzymes that cut much further away than 10 nucleotides, namely type
IIIS restriction enzymes, unfortunately do so in a less precise way, i.e. the
cut distance may vary by two nucleotides and the cut length may also vary by
about two nucleotides. However, to fitting sticky ends we need to cut very
precise.

\subsection{Example}
To illustrate this design, we give an example encoding of a program that
multiplies two powers of two. The three symbols, X, Y and Z can be encoded as

\begin{center}
  \begin{tikzpicture}
    \node[mytext]  (letX)    {X:};
    \node[mytext, right of=letX]  (seqX)   {\ttfamily 5-AA-3\\\ttfamily 3-TT-5};
    \node[mytext, right of=seqX]  (letY)    {Y:};
    \node[mytext, right of=letY]  (seqY)   {\ttfamily 5-GG-3\\\ttfamily 3-CC-5};
    \node[mytext, right of=seqY]  (letZ)    {Z:};
    \node[mytext, right of=letZ]  (seqZ)   {\ttfamily 5-CC-3\\\ttfamily 3-GG-5};
  \end{tikzpicture}
\end{center}
Thus the initial input of a program that multiplies 2 by 2, XXYY, is

\begin{center}
  \begin{tikzpicture}
    \node[mytext] (seq) {
      \ttfamily 5-AATTTAATTTGGTTTGGTTT-3\\
      \ttfamily 3-\ \ \ \ ATTAAACCAAACCA\ \ -5
    };
  \end{tikzpicture}
\end{center}
The production rules are
\begin{center}
  \begin{tikzpicture}
    \node[mytext] (prodX) {X $\to$ X:};
    \node[mytext, right=of prodX] {
      \ttfamily 5-AATT\textcolor{cyan}{GCGATG}\textcolor{blue}{GGCGGA}\textcolor{magenta}{(N)}AAAAAAA-3\\
      \ttfamily 3-\ \ \ \ \textcolor{cyan}{CGCTAC}\textcolor{blue}{CCGCCT}\textcolor{magenta}{(N)}TTTTT\ \ -5
    };
    \node[mytext, below=of prodX] (prodY) {Y $\to$ YZ:};
    \node[mytext, right=of prodY] {
      \ttfamily 5-AACC\textcolor{cyan}{GCGATG}\textcolor{blue}{GGCGGA}\textcolor{magenta}{(N)}AAACCAAAGGAA-3\\
      \ttfamily 3-\ \ \ \ \textcolor{cyan}{CGCTAC}\textcolor{blue}{CCGCCT}\textcolor{magenta}{(N)}TTTGGTTTCC\ \ -5
    };
    \node[mytext, below=of prodY] (prodZ) {Z $\to$ ZZ:};
    \node[mytext, right=of prodZ] {
      \ttfamily 5-AAGG\textcolor{cyan}{GCGATG}\textcolor{blue}{GGCGGA}\textcolor{magenta}{(N)}AAAGGAAAGGAA-3\\
      \ttfamily 3-\ \ \ \ \textcolor{cyan}{CGCTAC}\textcolor{blue}{CCGCCT}\textcolor{magenta}{(N)}TTTCCTTTCC\ \ -5
    };
  \end{tikzpicture}
\end{center}
where the letters N correspond to padding (magenta), the cyan sequence is the
FokI target site, the blue sequence is the EciI target site, and the S represent
the possible terminal sequences.

%----------------------------------------------------------------------------------------
% RESULTS AND DISCUSSION
%----------------------------------------------------------------------------------------

\section{Discussion}

what else

%----------------------------------------------------------------------------------------
%	BIBLIOGRAPHY
%----------------------------------------------------------------------------------------

\renewcommand{\refname}{\spacedlowsmallcaps{References}} % For modifying the bibliography heading

\bibliographystyle{unsrt}

\bibliography{sample.bib} % The file containing the bibliography

%----------------------------------------------------------------------------------------

\end{document}
