%%%%%%%%%%%%%%%%%%%%%%%%%%%%%%%%%%%%%%%%%
% Arsclassica Article
% LaTeX Template
% Version 1.1 (1/8/17)
%
% This template has been downloaded from:
% http://www.LaTeXTemplates.com
%
% Original author:
% Lorenzo Pantieri (http://www.lorenzopantieri.net) with extensive modifications by:
% Vel (vel@latextemplates.com)
%
% License:
% CC BY-NC-SA 3.0 (http://creativecommons.org/licenses/by-nc-sa/3.0/)
%
%%%%%%%%%%%%%%%%%%%%%%%%%%%%%%%%%%%%%%%%%

%----------------------------------------------------------------------------------------
%	PACKAGES AND OTHER DOCUMENT CONFIGURATIONS
%----------------------------------------------------------------------------------------

\documentclass[
11pt, % Main document font size
a4paper, % Paper type, use 'letterpaper' for US Letter paper
oneside, % One page layout (no page indentation)
%twoside, % Two page layout (page indentation for binding and different headers)
headinclude,footinclude, % Extra spacing for the header and footer
BCOR5mm, % Binding correction
]{scrartcl}

\input{structure.tex} % Include the structure.tex file which specified the document structure and layout

\hyphenation{Fortran hy-phen-ation} % Specify custom hyphenation points in words with dashes where you would like hyphenation to occur, or alternatively, don't put any dashes in a word to stop hyphenation altogether

%----------------------------------------------------------------------------------------
%	TITLE AND AUTHOR(S)
%----------------------------------------------------------------------------------------

\title{\normalfont\spacedallcaps{Designing a Molecular 2-Tag System}} % The article title

%\subtitle{Subtitle} % Uncomment to display a subtitle

\author{\spacedlowsmallcaps{Nicolas Ochsner, Miroslav Phan}} % The article author(s) - author affiliations need to be specified in the AUTHOR AFFILIATIONS block

\date{} % An optional date to appear under the author(s)

%----------------------------------------------------------------------------------------

\begin{document}

%----------------------------------------------------------------------------------------
%	HEADERS
%----------------------------------------------------------------------------------------

\renewcommand{\sectionmark}[1]{\markright{\spacedlowsmallcaps{#1}}} % The header for all pages (oneside) or for even pages (twoside)
%\renewcommand{\subsectionmark}[1]{\markright{\thesubsection~#1}} % Uncomment when using the twoside option - this modifies the header on odd pages
\lehead{\mbox{\llap{\small\thepage\kern1em\color{halfgray} \vline}\color{halfgray}\hspace{0.5em}\rightmark\hfil}} % The header style

\pagestyle{scrheadings} % Enable the headers specified in this block

\tikzset{mytext/.style={rectangle, draw=none, align=center}}

%----------------------------------------------------------------------------------------
%	TABLE OF CONTENTS & LISTS OF FIGURES AND TABLES
%----------------------------------------------------------------------------------------

\maketitle % Print the title/author/date block

\setcounter{tocdepth}{2} % Set the depth of the table of contents to show sections and subsections only

\tableofcontents % Print the table of contents

\listoffigures % Print the list of figures

\listoftables % Print the list of tables

%----------------------------------------------------------------------------------------
%	ABSTRACT
%----------------------------------------------------------------------------------------

\section*{Abstract} % This section will not appear in the table of contents due to the star (\section*)

Models of computation are theoretical models, that are used to prove what kind
of computations are possible. One of the most studied models is the universal
Turing machine, which largely defines what computation means nowadays.
Another model, the 2 tag system, can be used to emulate a universal Turing
machine.
It works on a string by removing symbols from the front and conditionally
appending symbols at the end.
Design of a 2 tag system as a biomolecular process allows us to compute any
problem that can be solved by a universal Turing machine.
We propose an implementation which uses DNA ligation and restriction enzymes to
compute a program that is implemented in a set of engineered sequences.

%----------------------------------------------------------------------------------------
%	AUTHOR AFFILIATIONS
%----------------------------------------------------------------------------------------

%\let\thefootnote\relax\footnotetext{* \textit{Department of Biology, University of Examples, London, United Kingdom}}

%\let\thefootnote\relax\footnotetext{\textsuperscript{1} \textit{Department of Chemistry, University of Examples, London, United Kingdom}}

%----------------------------------------------------------------------------------------

\newpage % Start the article content on the second page, remove this if you have a longer abstract that goes onto the second page

%----------------------------------------------------------------------------------------
%	INTRODUCTION
%----------------------------------------------------------------------------------------

\section{Introduction}

A $m$ tag system is a computational model, that operates on a string of symbols.
At each step it reads the first symbol, removes $m$ symbols from the head of the
string and then appends a sequence of symbols based on the symbol it read to the
string.
Mathematically the tag system can be described by the number of symbols that
are removed $m$, the alphabet $\Sigma$ of symbols used and the production rules,
$\Sigma \to \Sigma*$.

% 2 powers of 2 example here.

In the following sections we will first propose a general outline of an
implementation of a 2 tag system and discuss it's features and limitations
based on the processes involved and earlier iterations (namely the iteration
we presented during the last lecture).

\section{Biological Tag System}

As an alphabet we use short sequences of DNA which encode a single symbol each.
The program is encoded through a set of dsDNA sequences, with sticky ends on
both sides, that encode the production rules.
The production rules consist of four parts.

\begin{enumerate}
  \item Recognition site: A sticky end that binds to the first symbol of the sequence
  \item FokI binding domains: Two FokI binding domains oriented in opposite directions and some padding
  \item Production sequence: The sequence of symbols that needs to be appended, based on the recognition site
  \item Terminal site: A sticky end for each possible symbol at the end of the current string.
\end{enumerate}

This also means, that for an alphabet with $n$ symbols, we need $n^2$ production
rule sequences.
There are $n$ production rules, which map the first symbol to a sequence, and
then $n$ possible symbols, we have to append the sequence to.
So for each of the $n$ production rules there are $n$ similar versions, that
just differ by the sticky end.

\subsection{Example}
To illustrate this design, we give an example encoding of a program that
multiplies two powers of two.
The three symbols, X, Y and Z can be encoded as

\begin{center}
\begin{tikzpicture}
  \node[mytext]  (letX)    {X:};
  \node[mytext, right of=letX]  (seqX)   {\ttfamily ACCT\\\ttfamily TGGA};
  \node[mytext, right of=seqX]  (letY)    {Y:};
  \node[mytext, right of=letY]  (seqY)   {\ttfamily CAAG\\\ttfamily GTTC};
  \node[mytext, right of=seqY]  (letZ)    {Z:};
  \node[mytext, right of=letZ]  (seqZ)   {\ttfamily TCGA\\\ttfamily AGCT};
\end{tikzpicture}
\end{center}
Thus the initial input of a program that multiplies 2 by 2, XXYY, is

\begin{center}
  \begin{tikzpicture}
    \node[mytext] (seq) {\ttfamily ACCTACCTCAAG....\\\ttfamily ....TGGAGTTCGTTC};
  \end{tikzpicture}
\end{center}
The production rules
\begin{center}
  \begin{tikzpicture}
    \node[mytext] {$X \to X\colon$};
  \end{tikzpicture}
\end{center}

%----------------------------------------------------------------------------------------
%	RESULTS AND DISCUSSION
%----------------------------------------------------------------------------------------

\section{Results and Discussion}

Reference to Figure~\vref{fig:gallery}. % The \vref command specifies the location of the reference

\begin{figure}[tb]
\centering
\includegraphics[width=0.5\columnwidth]{GalleriaStampe}
\caption[An example of a floating figure]{An example of a floating figure (a reproduction from the \emph{Gallery of prints}, M.~Escher,\index{Escher, M.~C.} from \url{http://www.mcescher.com/}).} % The text in the square bracket is the caption for the list of figures while the text in the curly brackets is the figure caption
\label{fig:gallery}
\end{figure}

\lipsum[10] % Dummy text

%------------------------------------------------

\subsection{Subsection}

\lipsum[11] % Dummy text

\subsubsection{Subsubsection}

\lipsum[12] % Dummy text

\begin{description}
\item[Word] Definition
\item[Concept] Explanation
\item[Idea] Text
\end{description}

\lipsum[12] % Dummy text

\begin{itemize}[noitemsep] % [noitemsep] removes whitespace between the items for a compact look
\item First item in a list
\item Second item in a list
\item Third item in a list
\end{itemize}

\subsubsection{Table}

\lipsum[13] % Dummy text

\begin{table}[hbt]
\caption{Table of Grades}
\centering
\begin{tabular}{llr}
\toprule
\multicolumn{2}{c}{Name} \\
\cmidrule(r){1-2}
First name & Last Name & Grade \\
\midrule
John & Doe & $7.5$ \\
Richard & Miles & $2$ \\
\bottomrule
\end{tabular}
\label{tab:label}
\end{table}

Reference to Table~\vref{tab:label}. % The \vref command specifies the location of the reference

%------------------------------------------------

\subsection{Figure Composed of Subfigures}

Reference the figure composed of multiple subfigures as Figure~\vref{fig:esempio}. Reference one of the subfigures as Figure~\vref{fig:ipsum}. % The \vref command specifies the location of the reference

\lipsum[15-18] % Dummy text

\begin{figure}[tb]
\centering
\subfloat[A city market.]{\includegraphics[width=.45\columnwidth]{Lorem}} \quad
\subfloat[Forest landscape.]{\includegraphics[width=.45\columnwidth]{Ipsum}\label{fig:ipsum}} \\
\subfloat[Mountain landscape.]{\includegraphics[width=.45\columnwidth]{Dolor}} \quad
\subfloat[A tile decoration.]{\includegraphics[width=.45\columnwidth]{Sit}}
\caption[A number of pictures.]{A number of pictures with no common theme.} % The text in the square bracket is the caption for the list of figures while the text in the curly brackets is the figure caption
\label{fig:esempio}
\end{figure}

%----------------------------------------------------------------------------------------
%	BIBLIOGRAPHY
%----------------------------------------------------------------------------------------

\renewcommand{\refname}{\spacedlowsmallcaps{References}} % For modifying the bibliography heading

\bibliographystyle{unsrt}

\bibliography{sample.bib} % The file containing the bibliography

%----------------------------------------------------------------------------------------

\end{document}
